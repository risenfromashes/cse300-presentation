\documentclass[9pt]{beamer}
\mode<presentation>

% Theme choice:
\usetheme{Madrid}%Darmstadt

\usecolortheme[RGB={0,100,50}]{structure}
 \usefonttheme{structurebold}
\setbeamercovered{invisible}
%\setbeamertemplate{navigation symbols}{}
\usepackage{dynblocks}
\usepackage{textpos}
\usepackage{amsfonts}
\usepackage{amsmath}
\usepackage{amssymb}
\usepackage{amsthm}
\usepackage{mathtools}
\usepackage{tikz}
\usetikzlibrary{positioning,calc}
\usepackage{hyperref}
\usepackage[utf8]{inputenc}
\usepackage[T1]{fontenc}

\usepackage[affil-it]{authblk}
\usepackage{etoolbox}
\usepackage{algorithm}
\usepackage{algpseudocode}
\usepackage{lmodern}
\usefonttheme{serif}
\usepackage{listings}


\makeatletter
\patchcmd{\@maketitle}{\LARGE \@title}{\fontsize{18}{19.2}\selectfont\@title}{}{}
\makeatother

\renewcommand\Authfont{\fontsize{12}{14.4}\selectfont}
\renewcommand\Affilfont{\fontsize{9}{10.8}\itshape}

% Title page details:
\title[Presentation]{Presentation} %add title




\begin{document}

\section{Recursive Formulation}

\tikzstyle{box}=[rectangle, draw=black,
           text centered, text=black, minimum width=1cm, minimum height=1cm]
\tikzstyle{mbox}=[rectangle, draw=blue, thick, fill=blue!5,
           text centered, text=black, minimum width=2cm, minimum height=1.25cm]
\begin{frame}
\centering
\Huge \textbf{ How to find derangements? }
\end{frame}

\begin{frame} {Recursive Formulation}
    \Large {\textbf{Base case ($n = 1$)}} \\
    \vspace{1cm}
    \begin{figure}
        \centering
        \begin{tikzpicture}
            \node[box, fill=blue!20] at (0,0) {$ a_1 $};
            \pause
            \node[mbox] at (0, -2) {$ !1 = 0$};
        \end{tikzpicture}
    \end{figure}
\end{frame}


\begin{frame} {Recursive Formulation}
    \Large {\textbf{Base case ($n = 2$)}} \\
    \vspace{1cm}
    \begin{figure}
        \centering
        \begin{tikzpicture}
            \node[box, fill=blue!20] at (0,0) {$ a_1 $};
            \node[box, fill=red!20] at (1.25,0) {$ a_2 $};
            \pause
            \node[box, fill=red!20] at (0,-2.5) {$ a_2 $};
            \node[box, fill=blue!20] at (1.25,-2.5) {$ a_1 $};

            \draw[->, thick] (0.62, -0.75) --++ (0, -1);
            \pause
            \node[mbox] at (5, -1.35) {$!2 = 1$};
        \end{tikzpicture}
    \end{figure}
\end{frame}


\begin{frame} {Recursive Formulation}
    \begin{block}{Recursive Case}
        For $n \ge 2$
    \end{block}
    \vspace{1cm}
    \begin{figure}
        \centering
        \begin{tikzpicture}
            \draw[fill=blue!10] (0.5,-1) rectangle (8, 1);
            \foreach \x in {1, 2, 3, 4}
            {
                \node[box, fill=green!10](a_\x) at (\x + \x * 0.25,0) {$ a_\x $};

            }
            \node at (6, 0) {...};
            \node[box, fill=green!10] (a_n_1) at (7 ,0) {$ a_{n-1} $};
            \node[box, fill=red!10] (a_n) at (9,0) {$ a_{n} $};
            \pause
            \draw[ ->] (a_n) edge [bend left] (a_1.south);
            \draw[ ->] (a_n) edge [bend left] (a_2.south);
            \draw[ ->] (a_n) edge [bend left] (a_3.south);
            \draw[ ->] (a_n) edge [bend left] (a_4.south);
            \draw[ ->] (a_n) edge [bend left] (a_n_1);

        \end{tikzpicture}
    \end{figure}
\end{frame}

\begin{frame} {Recursive Formulation}
    \begin{block} {Case 1}
        $a_1$ at $n$th position, $a_n$ at 1st position
    \end{block}

    \begin{figure}
        \centering
        \begin{tikzpicture}
            \uncover<4-> {\draw[fill=blue!10] (1.85,-1) rectangle (7.7, 1);}
            \foreach \x in {2, 3, 4}
            {
                \node[box, fill=green!10](a_\x) at (\x + \x * 0.25,0) {$ a_\x $};

            }
            \node at (6, 0) {...};
            \node[box, fill=green!10] (a_n_1) at (7,0) {$ a_{n-1} $};
            \only<1>{
                \node[box, fill=red!10] at (9,0) {$ a_{n} $};
            }
            \only<2->{
                \node[box, fill=green!10] (a_n) at (9,0) {$ a_{1} $};
            }
            \uncover<2-3> {
                \draw[ ->] (a_n) edge [bend left] (a_1.south);
            }
            \uncover<3> {
                \draw[ ->] (a_1.north) edge [bend left] (a_n);
            }
            \only<1>{
                \node[box, fill=green!10] at (1.25,0) {$ a_1 $};
            }
            \only<2->{
                \node[box, fill=red!10] (a_1) at (1.25,0) {$ a_n $};
            }
            \uncover<5-> {
                \node at (4.75, -1.5) {$!(n-2)$};
            }
        \end{tikzpicture}
    \end{figure}
\end{frame}

\begin{frame} {Recursive Formulation}
    \begin{block} {Case 2}
        $a_1$ at $n$th position, $a_n$ at $i$th position ($i \neq 1 $)
    \end{block}

    \begin{figure}
        \centering
        \begin{tikzpicture}
            \uncover<3->{
                \draw[fill=blue!10] (0.5,-1) rectangle (8, 1);
            }
            \foreach \x in { 2, 3, 4}
            {
                \node[box, fill=green!10](a_\x) at (\x + \x * 0.25,0) {$ a_\x $};

            }
            \node at (6, 0) {...};
            \node[box, fill=green!10] (a_n_1) at (7,0) {$ a_{n-1} $};
            \node[box, fill=green!10] (a_n) at (9,0) {$ a_{1} $};
            \node[box, fill=red!10] (a_1) at (1.25,0) {$ a_n $};
            \uncover<2> {
                \draw[ ->] (a_1) edge [bend right] (a_2.south);
                \draw[ ->] (a_1) edge [bend right] (a_3.south);
                \draw[ ->] (a_1) edge [bend right] (a_4.south);
                \draw[ ->] (a_1) edge [bend right] (a_n_1.south);
            }
            \uncover<4-> {
                \node at (4, -1.5) {$!(n-1)$};
            }
        \end{tikzpicture}
    \end{figure}
\end{frame}

\begin{frame} {Recursive Formulation}
    \begin{block}{Recursive Formula}
        \begin{align*}
            !n = \uncover<2-> {
                    \uncover<4> {
                        (n - 1) (
                    }
                    !(n-2)
                    \uncover<3-> {
                       \quad + \quad !(n-1)
                    }
                \uncover<4-> {
                    )
                }
            }
        \end{align*}
    \end{block}
    \vspace{1cm}
    \uncover<3-4>{
        \begin{figure}
        \centering
        \begin{tikzpicture}
            \draw[fill=blue!10] (0.5,-1) rectangle (8, 1);
            \foreach \x in {1, 2, 3, 4}
            {
                \node[box, fill=green!10](a_\x) at (\x + \x * 0.25,0) {$ a_\x $};

            }
            \node at (6, 0) {...};
            \node[box, fill=green!10] (a_n_1) at (7 ,0) {$ a_{n-1} $};
            \node[box, fill=red!10] (a_n) at (9,0) {$ a_{n} $};
            \pause
            \draw[ ->] (a_n) edge [bend left] (a_1.south);
            \draw[ ->] (a_n) edge [bend left] (a_2.south);
            \draw[ ->] (a_n) edge [bend left] (a_3.south);
            \draw[ ->] (a_n) edge [bend left] (a_4.south);
            \draw[ ->] (a_n) edge [bend left] (a_n_1);

        \end{tikzpicture}
        \end{figure}
    }

\end{frame}

\begin{frame}[fragile]{Recursive Function}
\begin{block}{C++ function}
\begin{lstlisting}[language=C++]
    int derangement(int n) {
        if(n == 1) return 0;
        if(n == 2) return 1;
        return (n - 1) * (derangement(n - 1)
                        + derangement(n - 2));
    }
\end{lstlisting}
\end{block}
\end{frame}




\end{document}
